\chapter{Code listing}

This is a representative sample of code for \jg. An attempt has been made to focus on interesting, relevant
code that serves an important function in the system.\begin{center}\rule[0.1cm]{0.6\textwidth}{0.5pt}\end{center}

\emph{The full code listing can be found in the \tt{src} directory on the CD.} 

\section{Main classes}
\subsection*{JazzGen.java}
This class generates the chords, bass and percussion, and sets up the \tt{GeneticGenerator} class to generate the main solo.
\lstinputlisting{CD/src/JazzGen.java}

\subsection*{GeneticGenerator.java}
This is the genetic algorithm. Triggered by JazzGen, it interfaces with many other classes to initialise the population,
select, mutate, crossover and evaluate the fitness of each generation. Finally, it returns a \tt{Phrase} object containing
the melody line.
\lstinputlisting{CD/src/GeneticGenerator.java}

\section{Genetic algorithm method classes}
\subsection*{CrossoverDoublePoint.java}
This is the basic double-point crossover used for most of the examples in this document.
\lstinputlisting{CD/src/CrossoverDoublePoint.java}

\subsection*{InitChaotic.java}
The chaotic initialiser, used with the Standard map. See also \tt{InitOneOverF.java} for the 1/f generation
used in most examples.
\lstinputlisting{CD/src/InitChaotic.java}

\subsection*{MutationFunction.java}
Contains all of the mutation methods, selected at random. User interface is handled by \tt{MutationFunctionUI.java}.
\lstinputlisting{CD/src/MutationFunction.java}

\section{Fitness modules}
\subsection*{FmKey}
A very simple fitness function (UI and other class methods removed) that simply returns the \% of notes that are
on-key.
\lstinputlisting{key.java}

\subsection*{FmExpectancy2}
Evaluates melodic expectancy according to Schellenberg's simplified model. \tt{regDirCE}, \tt{regRetCE} and \tt{proxCE} are weightings
for each of the three metrics involved.
\lstinputlisting{exp.java}

\subsection*{FmRepetition2}
Checks for similar melodic strings within a block, awarding partial scores for near-matches.
\lstinputlisting{rep2.java}

\section{Chord progression}
\subsection*{ChordProgression.java}
This larger class implements voice leading, chord progression parsing, and includes a test method to ensure
that both are correct.
\lstinputlisting{cp.java}

\chapter{Running the code}

A precompiled version can be found in the \tt{bin} directory; jMusic must first
be installed and added to the Java classpath. \jg\ can then be executed with \tt{java JazzGenUI}. Alternately
the source can be found in \tt{src}; important files from the \tt{bin} directory may need to be copied over
for execution:

\begin{description}
\item[lilypond.template] Required for Lilypond output.
\item[defaults.opt] Contains sensible options as derived in Chapter \ref{chap-testexp}, Testing and Experimentation.
\item[fitness.xlsx] Excel 2007 spreadsheet used for stat logging. After generating a phrase \tt{fitness.csv} will be
 created in the current directory. Copy this data into the first worksheet of \tt{fitness.xlsx} and check the other
 worksheets for graphs.
\end{description}

\section*{Usage information}

After setting the options, click \textbf{Generate}. Once the progress bar is full, click \textbf{Play} to hear
the composition. The user interface has been designed for personal experimentation only and may not be user
friendly to others.