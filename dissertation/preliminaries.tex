% declaration
  \thispagestyle{empty}
  \begin{spaced}{4em}
    \begin{center}
      \LARGE\textbf{Algorithmic composition of jazz}
    \end{center}
  \end{spaced}
  \begin{spaced}{3em}
    \begin{center}
      Submitted by: Richard Harris
    \end{center}
  \end{spaced}
  \begin{spaced}{5em}
    \section*{COPYRIGHT}

    Attention is drawn to the fact that copyright of this dissertation rests
    with its author. The Intellectual Property Rights of the products
    produced as part of the project belong to the University of Bath (see
    \url|http://www.bath.ac.uk/ordinances/#intelprop|).

    This copy of the dissertation has been supplied on condition that anyone
    who consults it is understood to recognise that its copyright rests with its
    author and that no quotation from the dissertation and no information
    derived from it may be published without the prior written consent of
    the author.

    \section*{Declaration}
    This dissertation is submitted to the University of Bath in accordance
    with the requirements of the degree of Bachelor of Science in the
    Department of Computer Science. No portion of the work in this dissertation
    has been submitted in support of an application for any other degree
    or qualification of this or any other university or institution of learning.
    Except where specifically acknowledged, it is the work of the author.
  \end{spaced}

  \begin{spaced}{5em}
    Signed:
  \end{spaced}

\newpage
\consultation{0}
\newpage
\abstract
\jg\ is a jazz improvisation system that uses a genetic algorithm to evolve a solo
and accompaniment over a user-supplied chord progression. To aid experimentation
the system is both modular and heavily configurable via an extensive user interface.
This report details the design, development and experimentation of \jg; various
musical rules and constraints are encoded into a fitness function with the aim of
producing ``good'' jazz music. The output is shown to be of generally acceptable quality
approaching that of a novice jazz player. In particular, Narmour's model of melodic
expectancy is applied to the generation with pleasing results, and repetition is
used to encourage structure. Output is still not close to most human improvisation, but
the system shows potential for further development and several extensions are
considered.
\newpage
\tableofcontents
\newpage
%\listoffigures
%\newpage
%\listoftables
%\newpage
%\lstlistoflistings

\chapter*{Acknowledgements}
First and foremost, my friends and family. My supervisor, Professor John Fitch,
who has provided support and feedback in a most wonderful manner. The holy
trinity of energy drinks: Relentless, Rockstar and Jolt Cola, for keeping me
alert and creative. Special mention must be made of Dr. Oetker's Ristorante
Speciale pizza\footnote{\url{http://www.oetker.co.uk/wga/oetker_uk/html/default/debi-5b2jwr.en.html}},
not because it was of particular help to this dissertation,
but because it is the best pizza ever created.
\newpage